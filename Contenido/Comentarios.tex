Hola Eduardo,

�Bienvenido a bordo en la realizaci�n de recursos! 

He revisado el primer borrador que has realizado y tengo algunas observaciones que dejarte:

1. Da saltos de l�nea, no uses \\ : 
La raz�n es muy sencilla, los archivos .tex son archivos de texto plano, si das un espacio o cinco, no ahorras nada. Adem�s, \\ es un comando muy diferente a dar dos saltos de l�nea. No est� hecho para separar p�rrafos.

2. Usa el entorno align en vez de la combinaci�n equation+aligned.

3. Debes redactar entre cada cuadro de texto. �Revisaste el m�dulo de probabilidad? Sino, ya te dejo aqu� una copia.

4. Debemos ser muy cuidadosos en la estructura del texto. Siempre se nos escapar�n cosas y por eso somos un equipo, pero es importante tenerlas en mente al redactar un texto. Piensa que est�s dando una clase o presentando un reporte a alguien que no entiende mucho. Va a ser necesario el uso de un lenguaje conciso y preciso, harto ejemplo y a veces hasta repetir cosas.
Nuevamente: revi m�sa eldulo de probabiliad. No tienes que leerlo todo (pues es largo), pero te puedes familiarizar con c�mo se redact� y para qu� se hizo una planilla tan extensa.

5. Si necesitas comandos, �a�adelos! Esto en el documento principal llamado: Calculo_I

6. Usa la segunda persona del singular: Debemos ser m�s c�lidos, es mejor decir un �tienes� a un �tenemos�. 
%%%%%%%% TERCERA PERSONA
\begin{example}
Resuelva la ecuaci�n \(ax^2+bx+c=0\) y reemplaze la soluci�n en \(f(x)=x^2\).

Para resolver esta ecuaci�n tenemos que usar la f�rmula cuadr�tica...
\end{example}

%%%%%%%% SEGUNDA PERSONA
\begin{example}
Resuelve la ecuaci�n \(a^x+bx+c=0\) y reemplaza la soluci�n en \(f(x)=x^2\).

Para resolver esta ecuaci�n tienes que usar la f�rmula cuadr�tica...
\end{example}


7. Usa \begin{example} y \begin{exaple} para ejercicios con retroalimentaci�n. No necesitamos usar \begin{proof} salvo en ejercicios te�ricos.

8. Separa, usa % a diestra y siniestra. 
%
% �Como si no existiera un ma�ana!
%
Esto har� que tu documento sea legible no s�lo para ti, sino para quienes lo vamos a revisar y editar.

9. Para gr�ficos usaremos GeoGebra. Cada archivo debe ser guardado. Si necesitas ayuda en esto, yo mismo editar�.

10. Eres libre de consultarme cualquier cosa que necesites.
%%%%%%%%%%%%%%%%%%%%%%%%%%%%%%%%%%%%%%%%%%%%%%%%%%%
